%%%%%%%%%%%%%%%%%%%%%%%%%%%%%%%%%%%%%%%%%
% Twenty Seconds Resume/CV
% LaTeX Template
% Version 1.0 (14/7/16)
%
% This template has been downloaded from:
% http://www.LaTeXTemplates.com
%
% Original author:
% Carmine Spagnuolo (cspagnuolo@unisa.it) with major modifications by
% Vel (vel@LaTeXTemplates.com)
%
% License:
% The MIT License (see included LICENSE file)
%
%%%%%%%%%%%%%%%%%%%%%%%%%%%%%%%%%%%%%%%%%

%----------------------------------------------------------------------------------------
%	PACKAGES AND OTHER DOCUMENT CONFIGURATIONS
%----------------------------------------------------------------------------------------

\documentclass[letterpaper]{twentysecondcv} % a4paper for A4

% Command for printing skill progress bars
\newcommand\skills{
~
  \smartdiagram[bubble diagram]{
        \textbf{Software}\\\textbf{Engineering},
        \textbf{Design}\\\textbf{Patterns},
        \textbf{Programming}\\\textbf{Paradigms},
        \textbf{Machine}\\\textbf{Learning},
        \textbf{Deep}\\\textbf{Learning},
        \textbf{Functional}\\\textbf{~~Programming~~}
    }
}

\interests{{French/4.5},{English/5},{Portuguese/6}}

%----------------------------------------------------------------------------------------
%  PERSONAL INFORMATION
%----------------------------------------------------------------------------------------

% If you don't need one or more of the below, just remove the content leaving the command, e.g. \cvnumberphone{}

\cvname{Willian \textbf{Ver Valem}} % Your name
\cvjobtitle{Lead Developer} % Job title/career

\cvlinkedin{10 av. Marly, Talence - France}
\cvnumberphone{+33 6 22 47 37 82} % Phone number
% \cvsite{www.vervalem.com} % Personal website
\cvmail{vervalempaiva@gmail.com} % Email address

%----------------------------------------------------------------------------------------

\begin{document}

\makeprofile % Print the sidebar


\section{Profile}

A father, a geek, a traveller, a ninja, a cook, a gamer... \\ ...who studied and
worked for several years in a technical field in Brazil, then moved to France
and shifted to computer science and obtained a Master’s degree in software
engineering. Efforts during the past year focused on machine/deep learning.
Qualities: autonomy, teamwork, decision-making, creativity, and responsibility
Reliable for project management, and high commitment in duty.





%----------------------------------------------------------------------------------------
%  EDUCATION
%----------------------------------------------------------------------------------------
\section{Education}

\begin{twenty} % Environment for a list with descriptions

  \twentyitem
      {2017-continued}
        {}
        { \textit{Deep Learning Foundations}}
        {\href{http://www.udacity.com/}{\textbf{Udacity}}}
        {e-learning}
        {This training covers many of today's deep learning techniques such as
          Convolutional Neural Networks, Recurent Neural Networks (LSTM),
          Generative Adversarial Neural Networks and Deep Reinforcement Learnig}
    \\
  \twentyitem
      {2017-2017}
        {}
        { \textit{Machine Learning Engineering}}
        {\href{http://www.udacity.com/}{\textbf{Udacity}}}
        {e-learning}
        {I acquired a new skill-set to deal with data such as \textit{supervised /
            unsupervised / reinforcement learning} and an initiation to \textit{deep learning}
          and got familiar with tools like \textit{Tensorflow, Keras, Scipy,...}}
    \\
  \twentyitem
      {2015 - 2017}
        {}
        { \textit{MSc., Software Engineer}}
        {\href{http://www.u-bordeaux.com/}{\textbf{University of Bordeaux}}}
        {Bordeaux, France}
        {I specialized in Software Engineering and Project Management. I learned SCRUM, Design Patterns, Data Mining, no-SQL
Data Bases...}
    \\
  \twentyitem
      {2013 - 2015}
    {}
        { \textit{BSc., Computer Science}}
        {\href{http://www.u-bordeaux.com/}{\textbf{University of Bordeaux}}}
        {Bordeaux, France}
        {My knowledge deepened learning Graph Theory, Operational Systems Programming (Unix kernel), language theory, network programming, languages such as : Java, Lisp, C, Bison, Yacc , Python ; and I had an initiation to robotics}
    \\
  % \twentyitem
  %     {2012 - 2013}
  %   {}
  %       {\textit{Technical Degree in Computer Science}}
  %       {\href{http://www.u-bordeaux.com/}{\textbf{University of Bordeaux}}}
  %       {Bordeaux, France}
  %       {I acquired the basics of programming and algorithms, project management and languages such as Java, C++}
  %\twentyitem{<dates>}{<title>}{<organization>}{<location>}{<description>}

\end{twenty}


%----------------------------------------------------------------------------------------
%  EXPERIENCE
%----------------------------------------------------------------------------------------



\section{Professional Experience}

\begin{twenty} % Environment for a list with descriptions

  \twentyitem
  {May 2017}
    {current}
        {\textit{Lead Developer}}
        {\href{http://www.projet-lucine.com/}{\textbf{Lucine}}}
        {}
        {
          {\begin{itemize}
            \item My responsibilities at \textit{Lucine} are: team and project management
              using agile (SCRUM) methodology and do the R\&D of the core AI
              system of the company, where I work mainly with \textit{Tensorflow},
              creating models for facial recognition of pain by using
              architectures that relies on convolutional and recurrent networks.
              Among my responsibilities are also tasks such as manage the
              company's resources on \textit{google cloud}, \textit{Vsphere} (HDS) server and some
              \textit{DevOps}.
          \end{itemize}}
        }
    \\


  \twentyitem
      {May 2013 -}
    {Sep 2013}
        {\textit{Internship in software development}}
        {\href{https://github.com/1flow/1flow}{\textbf{1flow}}}
        {}
        {
          {\begin{itemize}

            \item As a first experience, I did an internship in a start-up called \textit{1Flow}. I had to elaborate and develop a protocol for RSS flow and to create a full text parsing site of information ; I had to benchmark the different existing methods, and make choices, then develop the application and implement it for tests. I used \textit{Jenkins} setup and different ways of coding such as the \textit{Django framework} and \textit{MongoDB}.\\
        \item Illustration to be found at: \href{https://github.com/1flow/1flow}{github.com/1flow/1flow}
          \end{itemize}}
        }
    \\

\end{twenty}


% \section{Personal Training and Interests}
% \begin{itemize}
% \item Self-trained in development with frameworks like Tensorflow,
%   Keras, Scikit and tools like Git, emacs, vim, docker, kubernets.

% \item Udacity Machine Learning Nanodegree

% \item \textbf{And also...} bujinkan, fitness , RPG \& Tabletop games \\
%   without forgetting cooking, news, traveling
% \end{itemize}


\end{document}
